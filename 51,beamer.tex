\documentclass[8pt]{beamer}
\usetheme{Frankfurt}

\usepackage{hyperref}
\begin{document}
\begin{frame}
\begin{flushright}
THE LITERATURE REVIEW PROCESS IN E-RESEARCH
\hspace{6mm}
51
\vspace{5mm}
\end{flushright}

%paragraph 1(this is just comment)
model in this area or have any ideas why it is not appropriate?” This latter phrasing illustrates that you have done some research and thinking and may well be a useful contact and serious e-researcher.

\vspace{5mm}
\hspace{-2cm}
\textbf{Citing Net-based Resources in the Literature Review}
\vspace{3mm}

%paragraph 2(this is just comment)
There are a number of formats for referencing documents and correspondence obtained from the Internet. In general the format for most styles follows that prescribed for the referencing of paper based documents, with the addition of the Uniform Resource Locator (URL) and the date of access of the document appended to the end of the reference. For example, in American Psychology Association (APA) format
the equivalent paper reference is followed by the words:

\vspace{3mm}
\hspace{1cm}

%paragraph 3(this is just comment)
Retrieved on date from the World Wide Web:bttp://site address.

\vspace{3mm}
%paragraph 4(this is just comment)
Some citing guidelines (notably APA) do not encourage private emails, unarchived list postings,or postings to Usenet groups in the reference bibliography, because obtaining a copy of the corespondence may be difficult or impossible for the interested reader.
\end{frame}

\begin{frame}
Instead, these guidelines suggest referencing such private or difficult to retrieve material as "private email correspondence from name on date" or “posting to list name on date" within the text of the document. Other guides suggest that this information be kept and made available to the interested reader and that it be referenced in the bibliography in the format:
\vspace{3mm}
\hspace{1cm}

%paragraph 5(this is just comment)
Anderson, T.(16 September 2001). Subject:When will our book be published?[email to H. Kanuka], [Online]. Available email:heather.kanuka@ualberta.ca.
\vspace{5mm}

%paragraph 6(this is just comment)
For more information related to the format for citing electronic references,the World-Wide Web Virtual Library maintains a listing of sites entitled Electronic References Scbolarly Citations of Internet Sources at
\href{http://www.spaceless.com/WWWVL/.}{here}

%paragraph 7(this is just comment)
It is important to learn and consistently use the format in which your e-research results will eventually be published. Making use of consistent notation of all relevant fields from the very beginning of the research process will save you a great deal of time
in the long run. To aid in this data organization process, a brief discussion of personal reference management software appears in the next section.

\vspace{3mm}
\hspace{-1cm}
\textbf{PLAGIARISM AND NETWORKED SOURCES}
\vspace{3mm}

%paragraph 8(this is just comment)
Most academic writing has liberal doses of direct quotations from the works of others.This practice lends authenticity to the literature review and, done properly, can even
enhance the readability of the literature review. However, it is imperative that the work of others be properly acknowledged. Even if a quotation is not used directly,
ideas that are paraphrased by the researcher need to be credited to the original source. Given the pervasiveness of ideas, papers, reference sources,and commentaries on the

\end{frame}









\end{document}