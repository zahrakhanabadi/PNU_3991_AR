\documentclass{article}
\begin{document}
\hspace{-2cm}
50
\hspace{1cm}
CHAPTER FOUR
\vspace{5mm}

%paragraph 1(this is just comment)
key word appended,if the name is very common) and find a Web site with relevant information, including the expert's email address and phone number. However, the novice e-researcher should be careful not to abuse this availability and should utilize such contacts only when other,less demanding forms of communication have been exhausted.Most experts whom you would like to reference in your literature review are very busy people-if they were not,you probably wouldn't be interested in their work. In addition,most experts write books,publish articles,and create Web sites so that you can gain access to their ideas and comments.Attempts by the novice e-researcher to short circuit the process and go directly and personally to the expert, without checking their public work, will likely be interpreted as bothersome and not be answered.

%paragraph 2(this is just comment)
As an example of an inappropriate request,we recently received an email from a graduate student studying in a foreign country. The email noted that the student had read one of our articles, liked it, and wondered if we could send more information.
Since we had no idea which article was read (we have been publishing for many years), we were not inclined to even answer the letter.Alternatively, legitimate, wellinformed,clearly written, and polite questions and concerns may not only be answered,
but may be appreciated and lead to further contacts with experts in the field.

%paragraph 3(this is just comment)
\vspace{5mm}
\hspace{-2cm}
\textbf{Filtering Messages for Others}
\vspace{5mm}

It is impossible to follow all of the discussion groups and Web sites that may have information relevant to your field of study. Thus, many successful e-researchers develop informal networks of friends and colleagues who filter relevant information
from their own explorations of the network and forward appropriate messages,links,or referrals to them. This filtering can become institutionalized as the researcher sets up a formal or informal mailing list for messages or references that contain information relevant to the members of the list. In the early days of networking,prior to the
Internet, this transporting of information between networks was referred to as porting and porters were celebrated as "Unsung heroes of the Network Nation!"(Masthead,Netweaver Magazine).

%paragraph 4(this is just comment)
\vspace{5mm}
\hspace{-2cm}
\textbf{Making Effective Use of the Informal Network Resources}
\vspace{5mm}

To maximize the effectiveness of an inquiry,an e-researcher must be careful to ask a question or request assistance in an appropriate manner. As in any conversation, the researcher must be sure to use a manner and tone that is polite,respectful, and appre-
ciative. In addition, e-researchers must insure that they have done their own literature review and research work before asking others to do it for them.

%paragraph 5(this is just comment)
For example,a question such as"Does anyone know anything about school dropout for a research paper I'm doing?” will likely not result in any assistance and will certainly let the members of the group know you have a great deal to learn about both
the subject and the etiquette of Net-based discussions. A refined request such as “Tinto's model of student dropout seems to be used often in postsecondary,but an ERIC search turns up only a single study in a secondary school context.Does anyone
on this list (Usenet group, or virtual conference) know of any work, using Tinto's 

\end{document}