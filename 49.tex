\documentclass{article}

\usepackage{hyperref}

\begin{document}
\begin{flushright} 
THE LITERATURE REVIEW PROCESS IN E-RESEARCH
\hspace{6mm}
49
\end{flushright} 
\vspace{5mm}

%paragraph 1(this is just comment)
groups' FAQs are linked through the Internet FAQ consortium at
\href{http://www.faqs.org/}{here}
These lists are designed to inform new users about frequently asked questions and to help prevent regular readers from rereading responses to questions that have been dealt with on many previous occasions.

%paragraph 2(this is just comment)
\begin{itemize}
\item 
Use quotes to reference only relevant material from a previous post to which you are responding. A brief quotation is useful to provide context; however, long inclusions of past comments only waste bandwidth and add to screen clutter.
\item 
Follow the discourse for a few weeks before posting comments or questions
yourself to insure that your particular question is relevant to the interests of the
list members.
\item 
If in doubt about the appropriateness of a potential posting, email it privately to
the list owner for feedback before posting it publicly.
\item 
The use of HTML coding and the addition of attachments to postings always add to the size of the message and may result in messages that cannot be read by all members of the list. A better solution is to post an announcement of the availability of the resource to the list, Usenet group, or virtual conference and to post longer or multimedia messages and files to a Web site where the interested reader can selectively retrieve them.
\item 
Create a separate file folder in your mailbox for information you are sent when first subscribing to a new list. These first subscriber information postings will tell you how to resign or suspend your membership in the list-information that may be relevant but very difficult to find when you wish to resign from the group or change your email address.
\end{itemize}

%paragraph 3(this is just comment)
\vspace{5mm}
\hspace{-2cm}
\textbf{Virtual Conferences}
\vspace{5mm}

The first “virtual conference" on the Internet was organized in 1992 for the International Council for Distance Education (Anderson and Mason, 1993). Subsequently, virtual conferences have proliferated and continue to provide a forum for professional development that is much more cost-effective and accessible than their face-to-face
equivalent (for example see http://www.rmrplc.com). Virtual conferences are timedelimited in that they use combinations of synchronous and asynchronous tools to support presentation and dialogue for a limited period of time and usually on a particular topic (Anderson, 1996). Like their face-to-face counterparts, virtual conferences usually include keynote presentations, promotional displays, and small group discussions.

\vspace{5mm}
\hspace{-2cm}
\textbf{Direct Email}
\vspace{5mm}

%paragraph 4(this is just comment)
Writing directly to an expert in the field may be a useful way for the e-researcher to gain invaluable access to the “informal network." The use of powerful search engines usually allows one to enter the expert's name (in quotations and possibly with a

\end{document}